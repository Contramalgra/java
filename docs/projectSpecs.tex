% Preamble. Don't worry about it.
\documentclass{article}
\usepackage{setspace}
\usepackage[utf8]{inputenc}
\onehalfspacing

\begin{document}

% --- TITLE PAGE ---
\title{Donnervögel Consulting \\ Streamlined Grading System}
\author{Markus Balaski \\ Graeme Smith \\ Jordan Toering \\ Stephen Laboucane \\ Ian Pun \\ Colin Woodbury \\ Chazz Young}
\date{\today}
\maketitle
\clearpage
% ------------------

% --- REVISION HISTORY ---
% Add more entries here as the Spec undergoes major changes.
% More recent entries come first.
\textbf{Revision History}\\
\begin{center}
  \begin{tabular}{l l l | c}
    Version & Date & Members & Changes\\
    \hline
    1.0 & 2014 Jan 28 & All Members & Initial Specifications Complete
  \end{tabular}
\end{center}
\clearpage
% ------------------------

% --- TABLE OF CONTENTS ---
\tableofcontents
\clearpage
% -------------------------

% Use this as a template. Section numbers, etc, is all handled automatically.
\part{Functional Requirements}
\section{Role - The Muffin Man}
\subsection{Making Muffins}
There is much to consider when making muffins.
\subsubsection{Choosing the Right Ingredients}
Any Muffin Man knows the right selection of bacon can make or break a good
muffin. Basically you can type whatever you want here and LaTeX will format it
nicely for you.
\subsubsection{Mixing}
Put your back into it.
\subsubsection{Into the Oven}
Preheat your oven to $\frac{1}{\pi}\sum_{i=0}^{\infty}e^i$ degrees.
\subsection{Sellings Muffins}
\subsubsection{Presentation}
Ugly looking muffins won't get bought.
\subsubsection{A Good Store Front}
People won't come into a shady looking place, no matter how good your muffins are.
\subsubsection{On Wit and Charm}
When all else fails, hope you rolled a good CHA stat, and be a gentleman.
% --- END OF TEMPLATE ---

\section{Role - The System Administrator}
The System Administrator is responsible for:
\subsection{Account Management}
\subsubsection{Account Creation}
The following fields are necessary:
\begin{itemize}
  \item The account owner's name
  \item A temporary password for the account owner
  \item The account type
\end{itemize}
\subsubsection{Existing Account Modification}
The System Admin can do:
\begin{itemize}
  \item Blocking or permission changing for accounts that are security risks.
  \item Password resets for users with forgotten passwords/multiple bad logins.
\end{itemize}
\subsection{System Log Management}
\subsection{System Backups/Restorations}
\subsection{Application Installation}

\section{Role - The Academic Administrator}
\subsection{The Power of the Academic Admin}
The Academic Admin does not have any jurisdiction over the role of System Administrator.
The Academic Administrator role assumes all functions of and can change/update any material
provided by the following users:
\begin{itemize}
  \item Assistant Academic Administrators
  \item Instructors
  \item TA/TMs
\end{itemize}

\section{Role - The Assistant Academic Administrator}
An Assistant Academic Administrator is able to create, modify, copy, and delete courses.
\subsection {Course management}
\subsubsection{Creating a Course}
To create a course, an Admin. Assistant must provide the following:
\begin{itemize}
	\item Name of course
	\item Course ID
	\item Assigned Teacher Assistants/Markers names
	\item Assigned Instructor name
	\item Start and end date of course
	\item List of activities in the course
	\item List of students enrolled in the course
\end{itemize}
\subsubsection{Modifying a Course}
An Admin. Assistant can to make any changes or additions to existing courses
freely after it has been created except for the rubric of an activity.
\subsubsection{Updating Student Lists}
An Admin. Assistant can fetch the latest list of students registered in a
specific course through the Course Management System.
When inputting the list of the students into the course, the course management
system will check for any updates from the previous list with the new one,
making any necessary additions or removal of students.
\subsubsection{Copying a Course}
An Admin. Assistant is allowed to copy a course. Copying a course will transcribe a new copy of these existing specifications:
\begin{itemize}
  \item Name of course
  \item Course ID
  \item List of activities including their individual:
    \begin{itemize}
    \item Marking rubric
    \item Language of the activity
    \item Type of activity
    \item Marking scheme (group/individual based)
    \item Due date
    \item Solution
    \end {itemize}
\end {itemize}
\subsubsection{Modifying a Rubric}
Modifying the rubric of an activity after it has been marked requires the whole
activity to be re-marked.
\subsubsection{Deleting a course}
An admin assistant is able to delete any existing course.

\section{Role - The Instructor}
The instructor has the main control over the contents of a course, its students, 
and its TAs or TMs. They can also grade any activities in the course.
\subsection{Activity Management}
Activity details include:
\begin {itemize}
	\item Name of activity
	\item Rubric of activity
	\item Type of activity
	\item Language of activity (programming or actual language)
	\item Group or individual activity
	\item Due date
	\item List of student grades
	\item Solution
\end {itemize}
\subsubsection{Create an activity}
Create an activity (assignment) and include the necessary details about it such 
as rubric and description.
\subsubsection{Modify an activity}
Edit any existing part of an activity, including its rubric. If the rubric is updated
the activity must be regraded fully.
\subsubsection{Copy activities}
Copy activities or a list of activities from a previous offering of the course, excluding
course offering specific info, but including details of the activity and its rubric.
\subsubsection{Enter tests/comparisons for code assignments}
Create tests or solutions for coding assignments that the students' work can be compared
to in grading.
\subsection{Student Management}
\subsubsection{Create student groups}
The instructor can create groups of students for an individual activity, or
multiple activities. 
\subsection{TA/TM Management}
\subsubsection{Assign TA/TMs to students}
The instructor can assign TAs and TMs to subsets of students or subsets of 
groups in the class for a particular activity or set of activities. The TA/TMs will then
only be able to grade those subsets for the activities the instructor chooses.
\subsection{Grade Assignments}
\subsubsection{View Assignments and Rubric}
A particular student's or group's assignment can be selected so the rubric is
shown side by side during grading.
\subsubsection{Enter Grades}
Markers are able to enter grades directly into the rubric while they are viewing
the assignment and rubric side by side.
\subsubsection{Add Comments to Student Work}
Markers are able to add comments directly into the pdf if it is a pdf submission,
or directly into the code if it is a code submission.
\subsubsection{Test Code Assignments}
The system will compile and run code assignments through the submitted tests,
then present the outputs side by side with the sample correct outputs.

\section{Role - The TA/TM Markers}
\subsection{Grade Assignments}
The same as the Grade Assignments section under the Instructor.
A TA/TM has can only access the assignments of the groups or students they
have been assigned to.

\section{The Login System}

\section{The Database}

\section{Logs}
There are two distinct types of logs: Academic Logs and System Logs.
\subsection{Academic Logs}
The academic log contains all records of additions and changes to the academic records.
The academic log documents grade and assignment modifications.  The academic log
is visible to every user excluding the Ta/Tm marker.
\subsection{System Logs}
The system log contains a record of the systems tasks.  It documents user sign-ins
and system backup times. The system log is only visible to the System Administrator.

% --------------------------------------------
\part{Prioritization of Function Requirements}
\section{Priority Lists}
\subsection{Core Features}
\begin{itemize}
  \item Account Creation
  \item Existing Account Modification
  \item The Power of the Academic Admin
  \item Creating a Course
  \item Modifying a Course
  \item Modifying a Rubric
  \item Create an Activity
  \item Modify an Activity
  \item Grade Assignments
  \item View Assignments and Rubric
  \item The Login System
\end{itemize}

\subsection{Most Important Features}
\begin{itemize}
  \item Copying a Course
  \item Copy Activities
  \item Enter tests/comparisons for code assignments
  \item Enter Grades
  \item Add Comments to Student Work
  \item Grade Assignments
\end{itemize}

\subsection{Less Important Features}
\begin{itemize}
  \item System Backups/Restorations
  \item Application Installation
  \item Create Student Groups
  \item Assign TA/TM's to students
  \item Test Code Assignments
  \item The Database
\end{itemize}

\subsection{Desired Features}
\begin{itemize}
  \item System Log Management
  \item Academic Logs
  \item System Logs
\end{itemize}

% --------------------------------
\part{Non-functional Requirements}
\section{Quality}
\subsection{Usability}
The application should be usable to all relevant university staff
regardless of computer experience.
\subsection{Accessibility}
\subsubsection{Valid Users}
Only users who have created an account with the System Administrator will be able
to use the application. Users will not be able to log in with their normal
university username and password.
\subsubsection{Availability Hours}
The system should be accessible at any time of day, except during backup times.
\subsection{Performance}
\subsubsection{User Base}
The client has specified that there will be around 200 user accounts. Ensure
support of up to \textbf{400 users.}
\subsubsection{Maximum Load}
The client has specified that up to 20 users might use the system during
peak times, thus the system should aim for no slow-down for up to \textbf{40 users.}
\subsection{Maintenance}
\subsubsection{Application Life Span}
The client has not indicated how long they plan to use the system once developed.
\subsubsection{Addition of Features}
The client has not requested a desire for the addition of new
features post release.

\section{Constraints}
\subsection{Platform}
The application will run on the Windows desktop.
\subsection{Implementation}
\subsubsection{Language}
The application will be written in Java using the Swing library for its GUI.
\subsubsection{Development Environment}
Eclipse?
\subsubsection{Version Control}
The source code will be managed via a software version control system.
\subsection{External Resources}
\subsubsection{SQL Database}
The application will store user and grading data on an MS SQL server located on
the university campus, and thus the user's computer must have a network connection.
\subsection{Licensing}
\subsubsection{Closed Source}
The application will be closed source under the Donnervögel Draconian License.
\subsubsection{GPL Incompatibility}
No libraries with GPL licensing will be included in the system.

\section{Other Requirements}
\subsection{System Administration}
\subsubsection{Account Management}
User accounts may only be created through the System Administrator.
\subsubsection{System Backups}
Database backups will be performed by the System Administrator and not handled
through the application.

% -------------------------------
\part{User Interface Description}

\section{General}
\subsection{Layout}
The application will be laid out in pages for each of the multiple tasks that
can be performed.
\subsection{Navigation}
Any page of the application will have a means to go back to the previous
page visited.

\section{Login}
\subsection{Login Screen}
\subsubsection{Inputs}
The login screen will contain inputs for "username" and "password", along with a
button to "enter" (confirm the username and password entered) and a button to handle
forgotten passwords. The username and password will be checked against the login database,
and the forgotten password will invoke some process to reset passwords.
\subsubsection{Outputs}
Upon hitting the "enter" button, the user will either be granted access or denied based
on an incorrect password. If a user is granted access, they will be sent
to a main UI screen offering their account's options in the system. If they enter a wrong
password, they will be told so, and given 5 opportunities to enter the correct password
before their account is locked. If the "forgot password" prompt is hit, the user will be sent
to a separate interface for resetting passwords
\subsection{Description}
The login will be the first screen shown upon opening the program. The login screen 
satisfies the separation of the functional requirements by role, as well as the capacity for users
to reset their passwords.

\section{Marking}
The marking system involves several stages each with its own screen in
the application.
\subsection{Course Choice}
The user is presented with a list of courses they are affiliated with. \\
Only courses in the current term will be shown. The courses themselves
will be buttons or clickable text.
\subsection{Assignment Choice by Student/Group}
The user can select two view styles for this screen. Note that in either case,
assignments not to be graded by the user will not be selectable. Assignments
will be buttons.
\subsubsection{Student View}
The user is presented with a grid of students and assignments. Only students
affiliated with the user will be visible.
\subsubsection{Group View}
The user is presented with a grid of groups and assignments. Only groups
affiliated with the user will be visible.
\subsection{Assignment Grading}
This screen shows all information relavent to the assignment, specifically:
\begin{itemize}
  \item Name of Activity
  \item Type of Activity
  \item Human Language of Activity/Instruction
  \item Programming Language of Activity (if applicable)
  \item Status as Group/Individual Activity (yes/no)
  \item Due date
\end{itemize}
\subsubsection{Rubric Display}
All entries for the rubric will be shown. Beside each entry will be an
area for the Marker to add a grade for that part of the rubric.\\
Total grade will be calculated based on the above entries and shown at the
bottom.
\subsubsection{Assignment Display}
Will the assignment be clearly visible right beside the rubric?
\subsubsection{Return Assignment Button}
There will be a button available to deliver a commented copy of the working
assignment back to the student.

\section{System Admin Maintenance Panels}

\section{Course Creation/Modification}
There will be 4 buttons available, \emph{Add a course}, \emph{Edit existing course},
\emph{Copy a course}, and \emph{Delete a course}.

\subsection{Adding a course}
There will be fields open to the user to input the neccessary information described
in (SECTION HERE): adding a course.

\subsection{Editing an existing course}
Ther will be a list of existing courses for you to choose from. Once the user has
picked the course to edit, there will be fields open to the user to make any
changes necessary. This satisfies req.
(SECTION HERE): editing existing course.

\subsection{Deleting an existing course}
There will be a list of existing courses for you to choose from with checkboxes
to satisfy req. (SECTION HERE) deleting a course.

\section{Activity Creation/Modification}
The view is virtually the same for both creation and modification.
The fields that will be shown for data entry or modification are:
\begin{itemize}
	\item Activity name
	\item Due date
	\item Type of activity
	\item Language of activity
	\item Size of group
	\item Rubric
	\begin{itemize}
		\item Number of expectations
		\item Description for each expectation
		\item Maximum grade for each expectation
	\end{itemize}
	\item Solution
	\item Test input and output files
\end{itemize}

Activities being created from scratch will have each field empty.
When editing, the current information is shown in the appropriate field.
Test input and output files are only an option for code activities.

\section{Test Suite}
\subsection{Test Suite Screen}
\subsubsection{Inputs}
The only inputs the test suite would take would be comments to place in the submitted code as well as the user input needed to interact with the console output of the submitted code.
\subsubsection{Outputs}
The test suite would output to the screen the reference code, the submitted code and the console messages.
\subsection{Description}
The test suite screen contains both the reference code and the student submitted code in a side by side format.  It includes a section for the marker to view and interact with the output of the submitted code.  The test suite also allows markers to add comments directly to the submitted code.  The test suite satisfies for all roles the functional requirement of a side-by-side code marking system, as well as marker interaction with the submitted code.

\section{Student/TA Group Management}

\end{document}
% Nothing past this will be included in the document.
